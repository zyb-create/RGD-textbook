% !TeX spellcheck = en_US


\chapter{Non-equilibrium Evaporation and Condensation}
\label{chap:Evap_Cond}


The fast spectral method developed in Chapter~\ref{chap:dense} is adopted to study the non-equilibrium evaporation and condensation at the liquid-vapor interface. 


\section{State of the problem}





\begin{figure}[h]
	\centering
	\includegraphics[width=0.8\textwidth]{Evaporation_demo}
	\caption{ \cite{Kon2014,Ohashi2020ScientificRep} (a) Schematic of the present simulation condition (steady vapor/gas–liquid equilibrium). At the center of the system, there is a thin liquid film; (b) Schematic of the molecular fluxes in the vicinity of vapor/gas–liquid interface. We can count the number of molecules using mixture gas and liquid boundaries; $J^\omega_{evap}$	denotes the evaporating molecular mass flux for $\omega$ component molecules, $J^\omega_{ref}$ the reflecting molecular mass flux, and $J^\omega_{cond}$ the condensing molecular mass flux.
	}
	\label{Evap_demo}
\end{figure}
\newpage

\section{Monatomic matter}


\subsection{Kinetic modeling beyond hard-sphere model}

\subsection{Evaporation and condensation coefficients}

\subsection{Extraction of kinetic boundary condition}

\newpage


\section{Polyatomic matter}


\subsection{Kinetic modeling}

\subsection{Evaporation and condensation coefficients}

\subsection{Extraction of kinetic boundary condition}


\newpage


\section{Multi-component evaporation and condensation}

\newpage 

\section{Moving contact line}
\newpage 




